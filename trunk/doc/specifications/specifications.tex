\documentclass{article}

\usepackage{amsmath}
\usepackage{xspace}
\usepackage{xcolor}
\usepackage{stmaryrd}
\usepackage{algorithm}
\usepackage{algorithmic}
\usepackage{listings}
\usepackage{supertabular}
\usepackage{multiletter}
\usepackage{morefloats}

\title{Specifications of Implemented Refactorings}
\author{Max Sch\"afer, Tom\'a\v{s} Ko\v{c}isk\'y}

\newcommand{\refactoring}[1]{\textsc{#1}}
\newcommand{\refactoringNoExt}[1]{\lfloor\refactoring{#1}\rfloor}
\newcommand{\type}[1]{\ensuremath{\text{\textsl{#1}}}}
\newcommand{\util}[1]{\ensuremath{\text{\texttt{#1}}}}
\newcommand{\orelse}{\,\textbf{or}\,}
\newcommand{\bind}{\gg=}
\newcommand{\assert}{\textbf{assert}\,\,}
\newcommand{\locked}[1]{\ensuremath{\llbracket #1\rrbracket}}
\newcommand{\option}[1]{\ensuremath{\text{\texttt{option}\xspace #1}}}
\newcommand{\None}{\util{None}\xspace}
\newcommand{\Some}[1]{\util{Some}\xspace #1}
\newcommand{\listtp}[1]{\ensuremath{\text{\texttt{list}\xspace #1}}}
\newcommand{\settp}[1]{\ensuremath{\text{\texttt{set}\xspace #1}}}
\newcommand{\sourcelink}[1]{\texttt{#1}}

\definecolor{KWColor}{rgb}{0.5,0,0.67}

\lstset{
  language=[JastAdd]Java,
  basicstyle=\ttfamily\small,
  commentstyle=\footnotesize\rmfamily\emph,
  keywordstyle=\bf\ttfamily\small\color{KWColor},
  morekeywords={with},
  escapeinside={/*@}{@*/},
  literate={[}{{$\lfloor$}}1 {]}{{$\rfloor$}}1,
}

\newcommand{\code}[1]{\lstinline$#1$}
\newcommand{\progoutput}[1]{\texttt{#1}}
\lstnewenvironment{java}{}{}
\newcommand{\kw}[1]{\textbf{\color{KWColor}{#1}}}

\begin{document}
\maketitle

This document collects the pseudo-code specifications of all refactoring implemented in our engine. \textbf{Note:} This is work in progress; some specifications are missing, and not all implementations agree completely with the specifications.

\section{Pseudocode Conventions}
We give our specifications in generic, imperative pseudocode. Parameters and return values are informally typed, with syntax tree nodes having one of the types from Fig.~\ref{fig:node types}. Additionally, we use an ML-like \util{option} type with constructors \util{None} and \util{Some} for functions that may or may not return a value.

Where convenient, we make use of ML-like lists, with list literals of the form $[1; 2; 3]$ and $|xs|$ indicating the length of list $xs$.

The names of refactorings are written in \refactoring{small caps}, whereas utility functions appear in \util{monospace}. A list of utility functions with brief descriptions is given in Fig.~\ref{fig:utility}. An invocation of a refactoring is written with floor-brackets $\refactoringNoExt{like this}()$ to indicate that any language extensions used in the output program produced by the refactoring should be eliminated before proceeding.

We write $A<:B$ to mean that type $A$ extends or implements type $B$, and $m<:m'$ to mean that method $m$ overrides method $m'$.

\section{The Refactorings}

\subsection{\refactoring{Convert Anonymous to Local}}
This refactoring converts an anonymous class to a local class. Implemented in \sourcelink{TypePromotion/AnonymousClassToLocalClass.jrag}; see Algorithm~\ref{alg:ConvertAnonymousToLocal}.

% There is makeFinal parameter in the implementation
% implementations doesn't use the refactorings mentioned.

\begin{algorithm}[p]
\caption{$\refactoring{Convert Anonymous to Local}(A : \type{AnonymousClass}, n : \type{Name}) : \type{LocalClass}$}\label{alg:ConvertAnonymousToLocal}
\begin{algorithmic}[1]
\REQUIRE Java
\ENSURE Java $\cup$ locked names
\medskip
\STATE $c \leftarrow \text{class instance expression containing $A$}$
\STATE $d \leftarrow \refactoringNoExt{Extract Temp}(c, \util{unCapitalise}(n))$
\STATE $b \leftarrow \text{enclosing body declaration of $d$}$
\STATE $\util{lockNames}(b, n)$
\STATE convert $A$ to class named $n$, remove it from $c$
\STATE $\refactoring{Insert Type}(b, A)$
\STATE lock type access of $c$ to $A$
\STATE $\refactoring{Inline Temp}(d)$
\RETURN $A$
\end{algorithmic}
\end{algorithm}

\subsection{\refactoring{Convert Anonymous to Nested}}
This refactoring converts an anonymous class to a member class. Implemented in \sourcelink{TypePromotion/AnonymousClassToMemberClass.jrag}; see Algorithm~\ref{alg:ConvertAnonymousToNested}.

% Still missing makeFinal, makePublic

\begin{algorithm}[p]
\caption{$\refactoring{Convert Anonymous to Nested}(A : \type{AnonymousClass}, n : \type{Name}) : \type{MemberType}$}\label{alg:ConvertAnonymousToNested}.
\begin{algorithmic}[1]
\REQUIRE Java
\ENSURE Java
\medskip
\STATE $L \leftarrow \refactoring{Convert Anonymous to Local}(A, n)$
\RETURN $\refactoring{Convert Local to Member Class}(L)$
\end{algorithmic}
\end{algorithm}

Note: the implementation additionally handles the case where $A$ occurs in a field initialiser.

\subsection{\refactoring{Convert Local to Member Class}}
This refactoring converts a local class to a member class. Implemented in \sourcelink{TypePromotion/LocalClassToMemberClass.jrag}; see Algorithms~\ref{alg:ConvertLocalToMemberClass},~%
\ref{alg:CloseOverVariables}.
%\ref{alg:CloseOverTypeVariables}, \ref{alg:CloseOverLocalVariables}.

\begin{algorithm}[p]
\caption{$\refactoring{Convert Local to Member Class}(L : \type{LocalClass}) : \type{MemberType}$}\label{alg:ConvertLocalToMemberClass}
\begin{algorithmic}[1]
\REQUIRE Java
\ENSURE Java $\cup$ locked names, fresh variables
\medskip
\STATE $A \leftarrow \text{enclosing type of $L$}$
\STATE $\util{closeOverTypeVariables}(L)$
\STATE $\util{closeOverLocalVariables}(L)$
\IF{$L$ is in static context}
  \STATE make $L$ static
\ENDIF
\STATE $\util{lockNames}(\util{name}(L))$
\STATE lock all names in $L$
\STATE remove $L$ from its declaring method
\STATE $\refactoring{Insert Type}(A, L)$
\end{algorithmic}
\end{algorithm}

%\begin{algorithm}[p]
%\caption{$\util{closeOverTypeVariables}(L : \type{LocalClass})$}\label{alg:CloseOverTypeVariables}
%\begin{algorithmic}[1]
%\medskip
%\STATE $m \leftarrow \text{empty map}$
%\STATE $U \leftarrow \text{accesses to $L$}$
%\FORALL{accesses $V$ to type variables $T$ of the enclosing body declaration}
%  \IF{$m(T)$ undefined}
%    \STATE create new type variable $T'$ with same bounds as $T$
%    \STATE add $T'$ as type parameter to $L$
%    \STATE $m(T) \leftarrow T'$
%    \FORALL{$u\in U$}
%      \STATE add locked access to $T$ as type argument to $u$
%    \ENDFOR
%  \ENDIF
%  \STATE lock $V$ onto $m(T)$
%\ENDFOR
%\end{algorithmic}
%\end{algorithm}
%
%\begin{algorithm}[p]
%\caption{$\util{closeOverLocalVariables}(L : \type{LocalClass})$}\label{alg:CloseOverLocalVariables}
%\begin{algorithmic}[1]
%\medskip
%\STATE $m \leftarrow \text{empty map}$
%\FORALL{accesses $v$ to local variables $x$ of enclosing body declaration}
%  \IF{$m(x)$ undefined}
%    \STATE create \code{private final} field $f$ of same type as $x$
%    \STATE add $f$ to $L$
%    \STATE $m(v) \leftarrow f$
%    \FORALL{constructors $c$ of $L$}
%      \STATE create new parameter $p$ of same type and name as $x$
%      \STATE insert $p$ as first parameter of $c$
%      \IF{$c$ is chaining constructor}
%        \STATE add access to $p$ as parameter to chaining invocation
%      \ELSE
%        \STATE insert assignment from $p$ to $f$ as first statement in $c$
%      \ENDIF
%    \ENDFOR
%    \FORALL{instantiations $i$ of $L$}
%      \STATE insert access to $x$ as first argument to $i$
%    \ENDFOR
%  \ENDIF
%  \STATE lock $v$ onto $m(x)$
%\ENDFOR
%\end{algorithmic}
%\end{algorithm}


\subsection{\refactoring{Extract Class}}
This refactoring extracts some fields of a class into a newly created member class. Implemented in \sourcelink{ExtractClass/ExtractClass.jrag}; 
see Algorithms~\ref{alg:ExtractClass},~\ref{alg:MoveInitsTogether},~\ref{alg:PassInitsToConstructor}.

We can pass initializers to a constructor if they do not depend on values of previous initializers.

\begin{algorithm}[p]
\caption{$\refactoring{Extract Class}(C : \type{Class}, fs : \listtp{\type{Field}}, n : \type{Name}, fn : \type{Name})$}\label{alg:ExtractClass}
\begin{algorithmic}[1]
\REQUIRE Java
\ENSURE Java $\cup$ locked names, locked dataflow, first-class array init
\medskip
\STATE $v \leftarrow \text{maximum visibility of any of the $fs$}$
\STATE $W \leftarrow \text{new \code{static} class of name $n$ with visibility $v$}$
\STATE $\refactoring{Insert Type}(C, W)$
\STATE $w \leftarrow \text{new field of type $W$ and name $fn$, initialised to a new instance of $W$}$
\STATE $\refactoring{Insert Field}(C, w)$
\FORALL{$f\in fs$}
  \STATE \assert $f$ is not static
  \FORALL{uses $v$ of $f$}
    \STATE qualify $v$ with a locked access to $w$
  \ENDFOR
  \IF{$f$ has initialiser}
    \STATE split field declaration and initializer, leaving initializer in initializer block after
  \ENDIF
  \STATE remove $f$
  \STATE $\refactoring{Insert Field}(W, f)$
\ENDFOR
\STATE $inits \leftarrow \{\text{initializers of }fs\}$
\STATE $\refactoring{Move Inits Together}(inits, w)$ \orelse{} $\refactoring{Id}()$
\STATE merge consecutive $inits$ to common initializer blocks
\STATE $\refactoring{PassInitsToConstructor}(inits, w)$ \orelse{} $\refactoring{Id}()$
\end{algorithmic}
\end{algorithm}


\begin{algorithm}[p]
\caption{$\refactoring{Move Inits Together}(inits : \listtp{InitializerBlock}, after : \type{Field})$}\label{alg:MoveInitsTogether}
\begin{algorithmic}[1]
\REQUIRE Java
\ENSURE Java
\medskip
\FORALL{$init\in inits$}
  \STATE lock names and dataflow in $init$
  \STATE remove $init$ and insert it after already moved initializers (possibly $after$)
  \STATE unlock names and dataflow in $init$
\ENDFOR
\end{algorithmic}
\end{algorithm}

\begin{algorithm}[p]
\caption{$\refactoring{PassInitsToConstructor}(inits : \listtp{InitializerBlock}, w : Field)$}\label{alg:PassInitsToConstructor}
\begin{algorithmic}[1]
\STATE \assert all $inits$ are in one initializer block
\STATE $index\leftarrow$ possition of $w$
\STATE $vars\leftarrow$ [ ]
\FORALL{$init$ in $inits$}
  \STATE $v\leftarrow\refactoringNoExt{Extract Temp}(\text{left side of $init$}, \text{fresh name}, index)$ 
  \STATE $vars\leftarrow v : vars$
  \STATE $index\leftarrow index + 1$
\ENDFOR
\FORALL{$var$ in $vars$}
  \STATE $\refactoring{Inline Temp}(var)$
\ENDFOR
\STATE in $W$ create a constructor for initializing all fields
\STATE change the constructor call for $w$ to initialize the fields and remove $inits$
\end{algorithmic}
\end{algorithm}

This is only a bare-bones specification. The implementation additionally allows to encapsulate the extracted fields, and to move the wrapper class $W$ to the toplevel.

\subsection{\refactoring{Extract Constant}}
This refactoring extracts a constant expression into a field. Implemented in \sourcelink{ExtractTemp/ExtractConstant.jrag}; see Algorithm~\ref{alg:ExtractConstant}.

\begin{algorithm}[p]
\caption{$\refactoring{Extract Constant}(e : \type{Expr}, n : \type{Name})$}\label{alg:ExtractConstant}
\begin{algorithmic}[1]
\REQUIRE Java
\ENSURE Java $\cup$ locked names, locked dataflow
\medskip
\STATE \assert $e$ is extractible
\STATE $A \leftarrow \text{enclosing type of $e$}$
\STATE $t \leftarrow \text{effective type of $e$}$
\STATE $f \leftarrow$ new \code{private} (\code{public} if $A$ is an interface) \code{static final} field of type $t$ and name $n$
\STATE $\refactoring{Insert Field}(A, f)$
\STATE lock names, flow, and synchronisation of $e$
\STATE set initialiser of $f$ to $e$
\STATE replace $e$ with locked access to $f$
\end{algorithmic}
\end{algorithm}

An expression is extractible if its type is not \code{void}, it is not a reference to a type or package, and it is not the keyword \code{super}; furthermore, it cannot be on the right-hand side of a dot.

The \emph{effective type} of an expression $e$ is the same as the type of $e$, except when the type of $e$ is an anonymous class, in which case the effective type is its superclass, or when the type of $e$ is a captured type variable, in which case the effective type is its upper bound.

\subsection{\refactoring{Extract Method}}
%See ECOOP 2009 publication. and the thesis
Implemented in \sourcelink{ExtractMethod/ExtractMethod.jrag}; 
see Algorithms \ref{alg:ExtractMethod},~\ref{alg:ExtractBlock}, \ref{alg:IntroduceAnonymousMethod}, %
\ref{alg:CloseOverVariables}, %
\ref{alg:EliminateReferenceParameters}, \ref{alg:LiftAnonymousMethod}.

\begin{algorithm}[p]
\caption{$\refactoring{Extract Method}(b \colon \type{Block}, i \colon \type{nat}, j \colon \type{nat}, n \colon \type{ident}) \colon \type{Method}$}
\label{alg:ExtractMethod}
\begin{algorithmic}[1]
\REQUIRE
\ENSURE
\medskip
\STATE $b' \leftarrow \lfloor\refactoring{Extract Block}\rfloor(b, i, j)$
\STATE $a \leftarrow \refactoring{Introduce Anonymous Method}(b')$
\STATE $\refactoring{Close over Variables}(a)$
\STATE $\refactoring{Eliminate Reference Parameters}(a)$
\RETURN $\lfloor\refactoring{Lift Anonymous Method}\rfloor(n, a)$
\end{algorithmic}
\end{algorithm}


\begin{algorithm}[p]
\caption{$\refactoring{Extract Block}(b \colon \type{Block}, i \colon \type{nat}, j \colon \type{nat}) \colon \type{Block}$}
\label{alg:ExtractBlock}
\begin{algorithmic}[1]
\REQUIRE no compound declarations
\ENSURE locked names
\medskip
\STATE $[s_0;\ldots;s_{n-1}] \leftarrow \text{statements in $b$}$
\STATE \assert $0\leq i\leq j<n$
\STATE lock all variable and type names in $b$
\FORALL{$i\leq k\leq j$}
  \STATE \assert $s_k$ is not a \code{case} or \code{default}
  \IF{$s_k$ declares a variable referenced after $s_j$}
    \STATE $\refactoring{Split Declaration}(s_k)$
    \STATE move $s_k$ before $s_i$
  \ENDIF
\ENDFOR
\STATE $b' \leftarrow \text{new block with statements $s_i,\ldots,s_j$}$
\STATE set statements of $b$ to $s_0,\ldots,s_{i-1},b',s_{j+1},\ldots,s_{n-1}$
\RETURN $b'$
\end{algorithmic}
\end{algorithm}

\begin{algorithm}[p]
\caption{$\refactoring{Introduce Anonymous Method}(b \colon \type{Block}) \colon \type{AnonymousMethod}$}
\label{alg:IntroduceAnonymousMethod}
\begin{algorithmic}[1]
\REQUIRE
\ENSURE locked control flow, locked names, \code{return void}, anonymous methods
\medskip
\STATE lock control flow successors in $b$
\STATE $[e_1;\ldots;e_n] \leftarrow$ locked accesses to all uncaught checked exceptions thrown in $b$
\IF{$b$ can complete normally}
  \STATE $c \leftarrow \mathtt{(()} \colon \text{\code{void throws}} \>\>e_1,\ldots,e_n \Rightarrow b \mathtt{)()}$
  \STATE replace $b$ with $c$\code{;}
\ELSE
  \IF{$b$ is in a method $m$}
    \STATE $T \leftarrow \text{locked access to return type of $m$}$
  \ELSE
    \STATE $T \leftarrow \text{\code{void}}$
  \ENDIF
  \STATE $c \leftarrow \mathtt{(()} \colon T \>\> \text{\code{throws}} \>\>e_1,\ldots,e_n \Rightarrow b \mathtt{)()}$
  \STATE replace $b$ with \code{return}\xspace$c$\code{;}
\ENDIF
\RETURN $c$
\end{algorithmic}
\end{algorithm}


\begin{algorithm}[p]
\caption{$\refactoring{Close over Variables}(a \colon \type{AnonymousMethod})$}
\label{alg:CloseOverVariables}
\begin{algorithmic}[1]
\REQUIRE
\ENSURE anonymous methods, \code{out} and \code{ref} parameters,
locked names
\medskip
\STATE $m \leftarrow \text{body declaration enclosing $a$}$
\STATE $V \leftarrow \emptyset; \mathit{Val} \leftarrow \emptyset;
\mathit{Out} \leftarrow \emptyset; \mathit{Ref} \leftarrow \emptyset$
\FORALL{variable accesses $va$ in $a$}
  \STATE $v \leftarrow \text{variable $va$ binds to}$
  \STATE \assert if $va$ is a write, then $v$ is not \code{final}
  \IF{$v$ is a local variable or parameter of $m$}
    \STATE $V \leftarrow V\cup\{v\}$
    \IF{$va$ has an incoming data flow edge}
      \IF{$v\in \mathit{Out}$}
        \STATE $\mathit{Out} \leftarrow \mathit{Out}\setminus\{v\}$
        \STATE $\mathit{Ref} \leftarrow \mathit{Ref}\cup\{v\}$
      \ELSIF{$v\not\in \mathit{Ref}$}
        \STATE $\mathit{Val} \leftarrow \mathit{Val}\cup\{v\}$
      \ENDIF
    \ENDIF
    \IF{$va$ has an outgoing data flow edge}
      \IF{$v\in \mathit{Val}$}
        \STATE $\mathit{Val} \leftarrow \mathit{Val}\setminus\{v\}$
        \STATE $\mathit{Ref} \leftarrow \mathit{Ref}\cup\{v\}$
      \ELSIF{$v\not\in \mathit{Ref}$}
        \STATE $\mathit{Out} \leftarrow \mathit{Out}\cup\{v\}$
      \ENDIF
    \ENDIF
  \ENDIF
\ENDFOR
\FORALL{$v\in V$}
  \IF{$v\in \mathit{Val}\cup \mathit{Out}\cup \mathit{Ref}$}
    \STATE $p \leftarrow \text{new parameter with same name and type as $v$}$
    \STATE make $p$ \code{ref} if $v\in \mathit{Ref}$, \code{out} if $v\in \mathit{Out}$
    \STATE add $p$ as parameter to $a$
    \STATE add access to $v$ as argument to $a$
  \ELSE
    \STATE $v' \leftarrow \text{new local variable with same name and type as $v$}$
    \STATE add $v'$ as local variable to $a$
  \ENDIF
\ENDFOR
\FORALL{type parameters $V$ of $m$ used in $a$}
  \STATE add type parameter $V'$ with same name and bounds as $V$ to $a$
  \STATE add type argument $V$ to $a$
\ENDFOR
\end{algorithmic}
\end{algorithm}



\begin{algorithm}[p]
\caption{$\refactoring{Eliminate Reference Parameters}(a \colon \type{AnonymousMethod}) \colon \type{AnonymousMethod}$}
\label{alg:EliminateReferenceParameters}
\begin{algorithmic}[1]
\REQUIRE no implicit \code{return}
\ENSURE anonymous methods
\medskip
\IF{$a$ has \code{ref} or \code{out} parameters}
  \STATE \assert $a$ has a single \code{ref} or \code{out} parameter
  \STATE \assert return type of $a$ is \code{void}
  \STATE $x \leftarrow \text{the \code{ref} or \code{out} parameter of $a$}$
  \STATE $v \leftarrow \text{the variable access passed as argument into $x$}$
  \STATE replace $a$ by $v\,$\code{=}$\,a$
  \STATE set return type of $a$ to type of $x$
  \STATE replace every \code{return;} statement with \code{return}$\,\,x$\code{;}
  \IF{$x$ is a \code{ref} parameter or it is live at the entry of $a$}
    \STATE make $x$ a value parameter
  \ELSE
    \STATE make $x$ a local variable
    \STATE remove argument $v$
  \ENDIF
\ENDIF
\RETURN $a$
% TODO: need to check that there is no outgoing data flow through exception;
\end{algorithmic}
\end{algorithm}

\begin{algorithm}[p]
\caption{$\refactoring{Lift Anonymous Method}(n \colon \type{ident}, a \colon \type{AnonymousMethod}) \colon \type{Method}$}
\label{alg:LiftAnonymousMethod}
\begin{algorithmic}[1]
\REQUIRE
\ENSURE locked names
\medskip
\STATE \assert $a$ does not reference any local variables from surrounding body declaration
\STATE \assert $a$ has no \code{ref} or \code{out} parameters
\STATE lock all names in $a$
\STATE $\overline{e} \leftarrow \text{argument list of $a$}$
\STATE $m \leftarrow \text{turn $a$ into method named $n$}$
\STATE make $m$ \code{static} if $a$ occurs in static context
\STATE $T \leftarrow \text{innermost type surrounding $a$}$
\STATE \assert $T$ has no member method with same signature as $m$
\STATE \assert no subtype of $T$ declares a method that would override
or hide $m$
\STATE lock all calls to methods named $n$
\STATE insert $m$ into $T$
\STATE $c \leftarrow \text{locked call of $m$ on arguments $\overline{e}$}$
\STATE replace $a$ with $c$
\RETURN $m$
\end{algorithmic}
\end{algorithm}




\subsection{\refactoring{Extract Temp}}
This refactoring extracts an expression into a local variable. Implemented in \sourcelink{ExtractTemp/ExtractTemp.jrag}; see Algorithms~\ref{alg:ExtractTemp}, \ref{alg:InsertLocalVariable}, \ref{alg:ExtractAssignment}, \ref{alg:MergeVariableDeclaration}.

\begin{algorithm}[p]
\caption{$\refactoring{Extract Temp}(e : \type{Expr}, n : \type{Name}) : LocalVar$}
\label{alg:ExtractTemp}
\begin{algorithmic}[1]
\REQUIRE Java
\ENSURE Java $\cup$ locked names, locked dataflow
\medskip
\STATE $t \leftarrow \text{effective type of $e$}$
\STATE $v \leftarrow \text{new local variable of type $t$ and name $n$}$
\STATE $s \leftarrow \text{enclosing statement of $e$}$
\STATE $\refactoring{Insert Local Variable}(s, v)$
\STATE $\refactoring{Extract Assignment}(v, e)$
\STATE $\refactoring{Merge Declaration}(v)$
\RETURN $v$
\end{algorithmic}
\end{algorithm}

\subsubsection{\refactoring{Insert Local Variable}}
The refactoring inserts a local variable before a given statement. Implemented in \sourcelink{ExtractTemp/IntroduceUnusedLocal.jrag}.

\begin{algorithm}[p]
\caption{$\refactoring{Insert Local Variable}(s : \type{Stmt}, v : \type{LocalVar}$)}
\label{alg:InsertLocalVariable}
\begin{algorithmic}[1]
\REQUIRE Java
\ENSURE Java $\cup$ locked names
\medskip
\STATE $b \leftarrow \text{enclosing block of $s$}$
\STATE \assert variable $v$ can be introduced into block $b$
\STATE $\util{lockNames}(b, n)$
\STATE insert $v$ before $s$
\end{algorithmic}
\end{algorithm}

\subsubsection{\refactoring{Extract Assignment}}
This refactoring extracts an expression into an assignment to a local variable. Implemented in \sourcelink{ExtractTemp/ExtractAssignment.jrag}.

\begin{algorithm}[p]
\caption{$\refactoring{Extract Assignment}(v : \type{LocalVar}, e : \type{Expr}) : \type{Assignment}$}
\label{alg:ExtractAssignment}
\begin{algorithmic}[1]
\REQUIRE Java
\ENSURE Java $\cup$ locked dependencies
\medskip
\STATE \assert $e$ is extractible
\STATE $a \leftarrow \text{new assignment from $e$ to $v$}$
\IF{$e$ is in expression statement}
  \STATE replace $e$ with $a$
\ELSE
  \STATE $s \leftarrow \text{enclosing statement of $e$}$
  \STATE lock all names in $e$
  \STATE insert $a$ before $s$
  \STATE replace $e$ with locked access to $v$
\ENDIF
\RETURN $a$
\end{algorithmic}
\end{algorithm}

\subsubsection{\refactoring{Merge Variable Declaration}}
This refactoring merges a variable declaration with the assignment immediately following it, if that assignment is an assignment to the same variable. Implemented in \sourcelink{ExtractTemp/MergeVarDecl.jrag}.

\begin{algorithm}[p]
\caption{$\refactoring{Merge Variable Declaration}(v : \type{LocalVar})$}
\label{alg:MergeVariableDeclaration}
\begin{algorithmic}[1]
\REQUIRE Java $\setminus$ multi-declarations
\ENSURE Java
\medskip
\IF{$v$ has initialiser}
  \RETURN
\ENDIF
\STATE $s \leftarrow \text{statement following v}$
\IF{$s$ is assignment to $v$}
  \STATE make RHS of $s$ the initialiser of $v$
  \STATE remove $s$
\ENDIF
\end{algorithmic}
\end{algorithm}

\subsection{\refactoring{Inline Constant}}
This refactoring inlines a constant field into all its uses. Implemented in \sourcelink{InlineTemp/InlineConstant.jrag}; see Algorithms~\ref{alg:InlineConstant-1}, \ref{alg:InlineConstant-2}, \ref{alg:RemoveField}.

\begin{algorithm}[p]
\caption{$\refactoring{Inline Constant}(f : \type{Field})$}
\label{alg:InlineConstant-1}
\begin{algorithmic}[1]
\REQUIRE Java $\setminus$ implicit assignment conversion
\ENSURE Java
\medskip
\FORALL{uses $u$ of $f$}
  \STATE $\refactoring{Inline Constant}(u)$
\ENDFOR
\STATE $\refactoring{Remove Field}(f)$
\end{algorithmic}
\end{algorithm}

\begin{algorithm}[p]
\caption{$\refactoring{Inline Constant}(u : \type{FieldAccess})$}
\label{alg:InlineConstant-2}
\begin{algorithmic}[1]
\REQUIRE Java
\ENSURE Java $\cup$ locked dependencies
\medskip
\STATE $f \leftarrow \text{field accessed by $u$}$
\STATE \assert $f$ is \code{final} and \code{static}, and has an initialiser
\STATE $e \leftarrow \text{locked copy of the initialiser of $f$}$
\STATE \assert if $u$ is qualified, then its qualifier is a pure expression
\STATE replace $u$ with $e$, discarding its qualifier if any
\end{algorithmic}
\end{algorithm}

\begin{algorithm}[p]
\caption{$\refactoring{Remove Field}(f : \type{Field})$}
\label{alg:RemoveField}
\begin{algorithmic}[1]
\REQUIRE Java
\ENSURE Java
\medskip
\IF{$f$ is not used and if it has an initialiser, it is pure}
  \STATE remove $f$
\ENDIF
\end{algorithmic}
\end{algorithm}

\subsection{\refactoring{Inline Method}}
This refactoring is inverse of $\refactoring{Extract Method}$. Implemented in \sourcelink{InlineMethod/}; 
see Algorithms~\ref{alg:InlineMethod}, \ref{alg:InlineMethodAccess}, \ref{alg:InlineToAnonymousMethod}, %
\ref{alg:IntroduceOutParameter}, %
\ref{alg:OpenVariables}, \ref{alg:InlineAnonymousMethod}, \ref{alg:InlineBlock}.


\begin{algorithm}[p]
\caption{$\refactoring{Inline Method}(m \colon \type{Method})$}
\label{alg:InlineMethod}
\begin{algorithmic}[1]
\REQUIRE Java
\ENSURE Java $\cup$ fresh variables, \code{with} statement, locked names
\medskip
  \FORALL{$methosAccess$ in $\util{polyUses}(m)$} 
    \STATE $\refactoring{Inline Method Access}(methodAccess)$
  \ENDFOR
  \STATE $\refactoring{Remove Method}(m)$ \orelse\ $\refactoring{Id}()$
\end{algorithmic}
\end{algorithm}


\begin{algorithm}[p]
\caption{$\refactoring{Inline Method Access}(ma \colon \type{MethodAccess})$}
\label{alg:InlineMethodAccess}
\begin{algorithmic}[1]
\REQUIRE Java
\ENSURE Java $\cup$ fresh variables, \code{with} statement, locked names
\medskip
  \STATE $am \leftarrow \refactoring{Inline To Anonymous Method}(ma)$
  \STATE $\refactoring{Introduce Out Parameter}(am)$
  \STATE $\refactoring{Open Variables}(am)$
  \STATE $node \leftarrow \refactoring{Inline Anonymous Method}(am)$
  \IF[in particular, it does not have a label]{$node$ is a $\type{Block}$}
    \STATE $\refactoring{Inline Block}(node)$
  \ENDIF
\end{algorithmic}
\end{algorithm}


\begin{algorithm}[p]
\caption{$\refactoring{Inline To Anonymous Method}(am \colon \type{MethodAccess}) : \type{AnonymousMethod}$}
\label{alg:InlineToAnonymousMethod}
\begin{algorithmic}[1]
\REQUIRE Java $\setminus$ \code{synchronized} qualifier, implicit \code{this} qualification
\ENSURE Java $\cup$ \code{with} statement, locked names
\medskip
  \STATE $A \leftarrow \text{host class}$
  \STATE \assert $\util{target}(ma)$ is unambiguous
  \STATE $target \leftarrow \util{target}(ma)$
  \STATE \assert $target$ has a body
  \STATE lock names in $target$
  \STATE $am \leftarrow$ copy $target$ as anonymous method with arguments from $ma$
  \IF{$ma$ is a \code{super} call}
    \STATE \assert body of $am$ does not reference enclosing instances
    \STATE adjust virtual calls in body of $am$ to bind to corresponding methods of $A$
  \ELSIF{$ma$ is qualified}
    \STATE $q \leftarrow \text{qualifier}$
    \STATE add \code{with(q)} statement around the body of $am$
    \STATE replace qualifier and the access with $am$
  \ELSE
    \STATE replace $ma$ with $am$
  \ENDIF
  \RETURN $am$
\end{algorithmic}
\end{algorithm}

\begin{algorithm}[p]
\caption{$\refactoring{Introduce Out Parameter}(am \colon \type{AnonymousMethod})$}
\label{alg:IntroduceOutParameter}
\begin{algorithmic}[1]
\REQUIRE
\ENSURE adds fresh variables, locked names
\medskip
  \STATE $\util{eliminateVarargs}()$
  \STATE $parent \leftarrow \util{parent}(am)$
  \IF{$parent$ is simple assignment expression}
    \STATE \assert destination of $parent$ assignment is a variable
    \STATE $v \leftarrow $ destination variable of $parent$
    \STATE set return type of $am$ to \code{void}
    \STATE add new fresh paremeter to $am$ with \code{out} modifier, type locked to $\util{type}(v)$
    \STATE add new argument to $am$ locked to $\util{decl}(v)$
    \STATE change \code{return} statements to assignment to the parameter and simple \code{return}
    \STATE replace $parent$ with $am$
  \ENDIF
\end{algorithmic}
\end{algorithm}


\begin{algorithm}[p]
\caption{$\refactoring{Open Variables}(am \colon \type{AnonymousMethod})$}
\label{alg:OpenVariables}
\begin{algorithmic}[1]
\REQUIRE 
\ENSURE adds fresh variables, locked names 
\medskip
  \FORALL{$(par, arg)$ in $\util{reverse}( \util{zip}\ \util{params}(am)\ \util{args}(am))$}
    \IF{$par$ is \code{in} parameter}
      \STATE $newdecl \leftarrow$ new variable declaration initialized to $arg$ with locked names
      \STATE insert $newdecl$ at the beginning of block of $am$
    \ELSE[$par$ is \code{out} parameter]
      \STATE \assert $arg$ is a variable access
      \STATE lock all uses of $par$ to $\util{decl}(arg)$
    \ENDIF
    \STATE remove $par$, $arg$ from $am$
  \ENDFOR
\end{algorithmic}
\end{algorithm}


\begin{algorithm}[p]
\caption{$\refactoring{Inline Anonymous Method}(am \colon \type{AnonymousMethod}) : ASTNode$}
\label{alg:InlineAnonymousMethod}
\begin{algorithmic}[1]
\REQUIRE no implicit \code{return}, no \code{return void}
\ENSURE no explicit \code{return} 
\medskip
  \STATE \assert $am$ has no parameters
  \STATE \assert one of the following three conditions is \code{true}
  \IF{$am$ is the expression in an expression statement}
    \STATE $l \leftarrow $ fresh label usable for block of $am$
    \FORALL{$ret$ in $\util{returns}(am)$}
      \IF{$ret$ has a result}
        \IF{result of $ret$ is pure}
	  \STATE \COMMENT{as $am$ is in an expression statement the result can be discarted and not evaluated}
	\ELSE
	  \STATE add an evaluation of the result of $ret$ before $ret$
	\ENDIF
      \ENDIF
      \STATE replace $ret$ with a \code{break} statement with label $l$
    \ENDFOR
    \STATE replace $am$ with its block and remove useless \code{break}s
  \ELSIF{$am$ is an expression closure, i.e.\ body is only a return statement}
    \STATE replace $am$ with expression from the \code{return} statement
  \ELSIF{$am$ is the expression in a \code{return} statement}
    \STATE replace the outer \code{return} with block of $am$
  \ENDIF
  \RETURN the expression or statement we replaced with
\end{algorithmic}
\end{algorithm}


\begin{algorithm}[p]
\caption{$\refactoring{Inline Block}(b \colon \type{Block})$}
\label{alg:InlineBlock}
\begin{algorithmic}[1]
\REQUIRE 
\ENSURE add locked names
\medskip
  \STATE \assert $b$ is a statement in a block without a label
  \STATE $\util{lockAllNames}(\text{parent block of }b)$
  \FORALL{$stmt$ in $b$}
    \STATE remove $stmt$ from $b$ and put it just before $b$
  \ENDFOR
  \STATE remove $b$
\end{algorithmic}
\end{algorithm}



\subsection{\refactoring{Inline Temp}}
This refactoring inlines a local variable into all its uses. Implemented in \sourcelink{InlineTemp/InlineTemp.jrag}; see Algorithms~\ref{alg:InlineTemp}, \ref{alg:SplitDeclaration}, \ref{alg:InlineAssignment}, \ref{alg:RemoveDecl}.

\begin{algorithm}[p]
\caption{$\refactoring{Inline Temp}(d : \type{LocalVar})$}
\label{alg:InlineTemp}
\begin{algorithmic}[1]
\REQUIRE Java
\ENSURE Java
\medskip
\STATE $a \leftarrow \refactoringNoExt{Split Declaration}(d)$
\STATE $\refactoringNoExt{Inline Assignment}(a)$
\STATE $\refactoringNoExt{Remove Decl}(v)$
\end{algorithmic}
\end{algorithm}

\begin{algorithm}[p]
\caption{$\refactoring{Split Declaration}(d : \type{LocalVar}) : \option{\type{Assignment}}$}
\label{alg:SplitDeclaration}
\begin{algorithmic}[1]
\REQUIRE Java $\setminus$ compound declarations
\ENSURE Java $\cup$ locked names, first-class array init
\medskip
\IF{$d$ has initialiser}
  \STATE $x \leftarrow \text{variable declared in $d$}$
  \STATE $a \leftarrow \text{new assignment from initialiser of $d$ to $x$}$
  \STATE insert $a$ as statement after $d$
  \STATE remove initialiser of $d$
  \RETURN \Some{$a$}
\ELSE
  \RETURN \None
\ENDIF
\end{algorithmic}
\end{algorithm}

\begin{algorithm}[p]
\caption{$\refactoring{Inline Assignment}(a : \type{Assignment})$}
\label{alg:InlineAssignment}
\begin{algorithmic}[1]
\REQUIRE Java $\setminus$ implicit assignment conversion
\ENSURE Java $\cup$ locked dependencies
\medskip
\STATE $x \leftarrow \text{LHS of $a$}$
\STATE \assert $x$ refers to local variable
\STATE $U \leftarrow \text{all $u$ such that $a$ is a reaching definition of $u$}$
\FORALL{$u\in U$}
  \STATE \assert $a$ is the only reaching definition of $u$
  \STATE \assert $u$ is not an lvalue
  \STATE \assert $u,a$ are in same body declaration
  \STATE replace $u$ with a locked copy of the RHS of $a$
\ENDFOR
\IF{$U\neq\emptyset$}
  \STATE remove $a$
\ENDIF
\end{algorithmic}
\end{algorithm}

\begin{algorithm}[p]
\caption{$\refactoring{Remove Decl}(d : \type{LocalVar})$}
\label{alg:RemoveDecl}
\begin{algorithmic}[1]
\REQUIRE Java $\setminus$ compound declarations
\ENSURE Java
\medskip
\IF{$d$ is not used and has no initialiser}
  \STATE remove $d$
\ENDIF
\end{algorithmic}
\end{algorithm}

\subsection{\refactoring{Insert Method}}
This refactoring inserts a method into a type declaration. Implemented in \sourcelink{Move/InsertUnusedMethod.jrag}; see Algorithms~\ref{alg:InsertMethod},~\ref{alg:canIntroduceMethod},~\ref{alg:typesToMakeAbstract}.

\begin{algorithm}[p]
\caption{$\refactoring{Insert Method}(m : \type{Method}, T : \type{Type})$}\label{alg:InsertMethod}
\begin{algorithmic}[1]
\REQUIRE Java
\ENSURE Java $\cup$ locked method names
\medskip
  \STATE $\util{lockMethodNames}(\util{name}(m))$
  \STATE \assert $\util{canIntroduceMethod}(m, T)$
  \STATE \assert \NOT $\util{isDynamicallyCallable}(m)$ 
  \STATE \assert $\{\util{name}(td) | \type{TypeDecl}\ td\in\util{below}(m)\} 
  		$\\$\qquad\qquad\cap\  
		\{\util{name}(t) | t\text{ is enclosing type of }T \vee t = T\} = \emptyset$
  \STATE insert method $m$ into the type $T$
  \IF{$m$ is \code{abstract}}
     \FORALL{$type$ in $\util{typesToMakeAbstract}(m)$}
       \STATE $\refactoring{Make Type Abstract}(type)$
     \ENDFOR
  \ENDIF
\end{algorithmic}
\end{algorithm}


\begin{algorithm}[p]
\caption{$\util{canIntroduceMethod}(m : \type{Method}, T : \type{Type})$}\label{alg:canIntroduceMethod}
\begin{algorithmic}[1]
  \STATE \assert $m$ is not \code{static} \OR $T$ is not inner
  \STATE \assert there is no local method in $T$ with same signature errasure as $m$
  \STATE \assert if there are any like-named methods in superclasses, we must be able to override or hide them, and
  			similarly for subclasses
\end{algorithmic}
\end{algorithm}

\begin{algorithm}[p]
\caption{$\util{typesToMakeAbstract}(m : \type{Method}) : \settp{Type}$}\label{alg:typesToMakeAbstract}
\begin{algorithmic}[1]
  \STATE do DFS from $\util{hostType}(m)$ through child types \\
    but do not visit a type that declares a method that \emph{overrides} $m$ \\
    (in particular, visit a type in a different package, even if it can't override $m$)
  \RETURN set of all visited types
\end{algorithmic}
\end{algorithm}


\subsection{\refactoring{Introduce Factory}}
This refactoring introduces a static factory method as a replacement for a given constructor, and updates all uses of the constructor to use this method instead. Implemented in \sourcelink{IntroduceFactory/IntroduceFactory.jrag}; see Algorithm~\ref{alg:IntroduceFactory}

\begin{algorithm}[p]
\caption{$\refactoring{Introduce Factory}(cd : \type{ConstructorDecl})$}
\label{alg:IntroduceFactory}
\begin{algorithmic}[1]
\REQUIRE Java
\ENSURE Java $\cup$ locked names
\medskip
\STATE $f \leftarrow \text{static factory method for $cd$}$
\FORALL{uses $u$ of $cd$ and its parameterised copies}
  \IF{$u$ is a class instance expression without anonymous class and it is not in $f$}
    \STATE replace $u$ with a call to $f$
  \ENDIF
\ENDFOR
\end{algorithmic}
\end{algorithm}

We use \util{createFactoryMethod} (implemented in \sourcelink{util/ConstructorExt.jrag}) to create the factory method corresponding to constructor $cd$ and insert it into the host type of $cd$. The factory method has the same signature as $cd$, but it has its own copies of all type variables of the host type used in $cd$.

\subsection{\refactoring{Introduce Indirection}}
This refactoring creates a static method $m'$ in type $B$ that delegates to a method $m$ in type $A$. Implemented in \sourcelink{IntroduceIndirection/IntroduceIndirection.jrag}; see Algorithm~\ref{alg:IntroduceIndirection}.

\begin{algorithm}[p]
\caption{$\refactoring{Introduce Indirection}(m : \type{Method}, B : \type{ClassOrInterface})$}
\label{alg:IntroduceIndirection}
\begin{algorithmic}[1]
\REQUIRE Java
\ENSURE Java $\cup$ locked names, \code{return void}
\medskip
\STATE \assert $B$ is non-library
\STATE $fn \leftarrow \text{fresh method name}$
\STATE $m' \leftarrow \text{copy of $m$ with locked names and empty body}$
\STATE set name of $m'$ to $fn$
\STATE $xs \leftarrow \text{locked accesses to parameters of $m'$}$
\STATE set body of $m'$ to \code{return}\xspace$m$\code{(}$xs$\code{);}
\STATE $\refactoring{Insert Method}(\util{hostType}(m), m')$
\STATE $\refactoring{Make Method Static}(m')$
\STATE $\refactoring{Move Static Method}(m', B)$
\end{algorithmic}
\end{algorithm}

\subsection{\refactoring{Introduce Parameter}}
This refactoring turns an expression into a parameter of the surrounding method. Implemented in \sourcelink{ChangeMethodSignature/IntroduceParameter.jrag}; see Algorithm~\ref{alg:IntroduceParameter}.

\begin{algorithm}[p]
\caption{$\refactoring{Introduce Parameter}(e : \type{Expr}, n : \type{Name})$}
\label{alg:IntroduceParameter}
\begin{algorithmic}[1]
\REQUIRE Java
\ENSURE Java $\cup$ locked names
\medskip
\STATE \assert $n$ is a valid name
\STATE \assert $e$ is extractible and constant
\STATE \assert $e$ appears within a method $m$
\STATE \assert $m$ is not overridden by and does not override any other methods
\STATE \assert $m$ has no parameter or local variable $n$
\STATE $\util{lockMethodCalls}(\util{name}(m))$
\STATE $t \leftarrow \text{effective type of $e$}$
\STATE $p \leftarrow \text{new parameter of type $t$ and name $n$}$
\STATE insert $p$ as the first parameter of $m$
\STATE replace $e$ with locked access to $p$
\FORALL{calls $c$ to $m$}
  \STATE insert a locked copy of $e$ as first argument of $c$
\ENDFOR
\end{algorithmic}
\end{algorithm}

\subsection{\refactoring{Introduce Parameter Object}}
This refactoring wraps a set $P$ of parameters of a method $m$ into a single parameter $n$ of type $w$, where $w$ is a newly created wrapper class containing fields corresponding to all the parameters in $P$. Implemented in \sourcelink{IntroduceParameterObject/IntroduceParameterObject.jrag}; see Algorithm \ref{alg:IntroduceParameterObject}.

\begin{algorithm}[p]
\caption{$\refactoring{Introduce Parameter Object}(m : \type{Method}, P : \settp{\type{Parameter}}, w : \settp{Name}, n : \settp{Name})$}
\label{alg:IntroduceParameterObject}
\begin{algorithmic}[1]
\REQUIRE Java $\setminus$ variable arity parameters
\ENSURE Java $\cup$ locked names
\medskip
\STATE \assert $m$ has a body
\STATE \assert the parameters in $P$ are in contiguous positions $i, \ldots, i+k$
\STATE $W \leftarrow$ new class containing fields for all the $P$ and a standard constructor to initialise them
\STATE $\refactoring{Insert Type}(\util{hostType}(m), W)$
\STATE $\util{lockMethodCalls}(\util{name}(m))$
\FORALL{relatives $r$ of $m$}
  \STATE \assert $r$ has no parameter or local variable with name $n$
  \STATE $[p_1;\ldots;p_n] \leftarrow \text{parameters of $r$}$
  \STATE $p \leftarrow \text{new parameter of type $W$ and name $n$}$
  \STATE replace parameters $p_i, \ldots, p_{i+k}$ with $p$
  \FORALL{$j\in\{i, \ldots, i+k\}$}
    \STATE $v_j \leftarrow \text{new variable of same name, type, and finality as $p_j$}$
    \STATE insert assignment from $p.f_j$ to $v_j$ at beginning of $m$
  \ENDFOR
  \FORALL{calls $c$ to $r$}
    \STATE $[a_1;\ldots;a_n] \leftarrow \text{arguments of $c$}$
    \STATE replace arguments $a_i, \ldots, a_{i+k}$ with \code{new }$W$\code{(}$a_i$\code{,}\ldots\code{,}$a_{i+k}$\code{)}
  \ENDFOR
\ENDFOR
\end{algorithmic}
\end{algorithm}

Note that we need to perform the transformation for all relatives of $m$, \emph{i.e.} for all methods $r$ such that there exists a method $m'$ with $m<:^*m'$ and $r<:^*m'$. We also lock all calls to methods of the same as $m$ in the whole program; this ensures that if overloading resolution changes due to the transformation, the name binding framework will insert appropriate casts to rectify the situation.

Note: the implementation actually: eliminates variable arity parameter for this method and adjusts all calls; does not require $p_i$ to be contiguous and adds new argument at the beginning. (This can be unsound for parameters with side effects!!!)

\subsection{\refactoring{Make Method Static}}
This refactoring makes a method static. Implemented in \sourcelink{MakeMethodStatic/MakeMethodStatic.jrag}; see Algorithm~\ref{alg:MakeMethodStatic}.

\begin{algorithm}[p]
\caption{$\refactoring{Make Method Static}(m : \type{Method})$}\label{alg:MakeMethodStatic}
\begin{algorithmic}[1]
\REQUIRE Java
\ENSURE Java $\cup$ return \code{void}, fresh variables, \code{with} statement, locked names, demand \code{final} modifier
\medskip
  \STATE \assert $m$ has a body
  \STATE $newMethod\leftarrow \util{copy}(m)$
  \STATE $delegator\leftarrow m$
  \STATE $\util{lockMethodNames}(\util{name}(delegator))$
  \STATE add \code{static} modifier to $newMethod$
  \STATE add new parameter to $newMethod$ with fresh name, type locked to $\util{hostType}(m)$, and demand final
  \STATE put a \code{with} statement around the body of $newMethod$ mapping \code{this} to the new parameter
  \STATE $\refactoring{Close Over Variables}(newMethod)$
  \STATE change the block of $delegator$ method to a call to $newMethod$ \\
  	with \code{this} and parameters of $delegator$ as arguments
  \STATE $\refactoring{Insert Method}(\util{hostType}(delegator), newMethod)$
\end{algorithmic}
\end{algorithm}



\subsection{\refactoring{Move Inner To Toplevel}}
This refactoring converts a member type to a toplevel type. Implemented in \sourcelink{TypePromotion/MoveMemberTypeToToplevel.jrag}; see Algorithms~\ref{alg:MoveMemberTypeToToplevel}, \ref{alg:InsertType}, \ref{alg:MakeTypeStatic}.

\begin{algorithm}[p]
\caption{$\refactoring{Move Member Type to Toplevel}(M : \type{MemberType})$}
\label{alg:MoveMemberTypeToToplevel}
\begin{algorithmic}[1]
\REQUIRE Java
\ENSURE Java $\cup$ locked names
\medskip
\IF{$M$ is not static}
  \STATE $\refactoringNoExt{Make Type Static}(M)$
\ENDIF
\STATE $p \leftarrow \util{hostPkg}(M)$
\STATE lock all names in $M$
\STATE remove $M$ from its host type
\STATE $\refactoring{Insert Type}(p, M)$
\end{algorithmic}
\end{algorithm}

\begin{algorithm}[p]
\caption{$\refactoring{Insert Type}(p : \type{Package}, T : \type{ClassOrInterface})$}
\label{alg:InsertType}
\begin{algorithmic}[1]
\REQUIRE Java
\ENSURE Java $\cup$ locked names
\medskip
\STATE \assert no type or subpackage of same name as $T$ in $p$
\STATE $\util{lockNames}(\util{name}(T))$
\STATE remove modifiers \code{static}, \code{private}, \code{protected} from $T$
\STATE insert $T$ into $p$
\end{algorithmic}
\end{algorithm}

\begin{algorithm}[p]
\caption{$\refactoring{Make Type Static}(M : \type{MemberType})$}
\label{alg:MakeTypeStatic}
\begin{algorithmic}[1]
\REQUIRE Java
\ENSURE Java $\cup$ \code{with}, locked names
\medskip
\STATE $[A_n;\ldots;A_1] \leftarrow \text{enclosing types of $M$}$
\FORALL{$i\in\{1,\ldots,n\}$}
  \STATE $f \leftarrow \text{new field of type $A_i$ with name \texttt{this\$i}}$
  \STATE $\refactoring{Insert Field}(M, f)$
  \FORALL{constructors $c$ of $M$}
    \STATE $p \leftarrow \text{parameter of type $A_i$ with name \texttt{this\$i}}$
    \STATE \assert no parameter or variable \texttt{this\$i} in $c$
    \STATE insert $p$ as first parameter of $c$
    \IF{$c$ is chaining}
      \STATE add \texttt{this\$i} as first argument of chaining call
    \ELSE
      \STATE $a \leftarrow \text{new assignment of $p$ to $f$}$
      \STATE insert $a$ after \code{super} call
    \ENDIF
  \ENDFOR
\ENDFOR
\FORALL{constructors $c$ of $M$}
  \FORALL{non-chaining invocations $u$ of $c$}
    \STATE $es \leftarrow \text{enclosing instances of $u$}$
    \STATE \assert $|es|=n$
    \STATE insert $es$ as initial arguments to $u$
    \STATE discard qualifier of $u$, if any
  \ENDFOR
\ENDFOR
\IF{$M$ not in inner class}
  \STATE put modifier \code{static} on $M$
\ENDIF
\FORALL{non-\code{static} callables $m$ of $M$}
  \IF{$m$ has a body}
    \STATE surround body of $m$ by\\
           \code{with(this\$n, ..., this\$1, this) \{...\}}%\\
  \ENDIF
\ENDFOR
\end{algorithmic}
\end{algorithm}

\subsection{\refactoring{Move Instance Method}}
This refactoring moves a method into a variable, which is either a parameter of that method or an accessible field. Implemented in \sourcelink{Move/MoveMethod.jrag}; see Algorithm~\ref{alg:MoveMethod}.

\begin{algorithm}[p]
\caption{$\refactoring{Move Method}(m : \type{InstanceMethod}, v : \type{Variable})$}
\label{alg:MoveMethod}
\begin{algorithmic}[1]
\REQUIRE Java
\ENSURE Java $\cup$ locked names, \code{return void}, fresh variables, demand \code{final}
\medskip
\STATE \assert $v$ is either a parameter of $m$ or a field
\STATE $T\leftarrow\text{type of $v$}$
\STATE \assert $T$ is a non-library class
\STATE \assert $m$ has a body and is not from library
\STATE $m'\leftarrow\text{copy of $m$ with \code{synchronized} removed and all names locked}$
\STATE $xs\leftarrow\text{list of locked accesses to parameters of $m$}$
\IF{$v$ is a parameter}
  \STATE $i\leftarrow\text{position of $v$ in parameter list of $m$}$
  \STATE remove $i$th parameter from $m'$
  \STATE remove $i$th element of $xs$
\ELSE
  \STATE $i\leftarrow 0$
\ENDIF
\STATE $v'\leftarrow\text{\code{final} local variable declaration with same name and type as $v$, initialised to \code{this}}$
\STATE insert $v'$ as first statement into $m'$
\STATE lock all uses of $v$ inside $m'$ to $v'$
\STATE $qs\leftarrow[]$
\FORALL{enclosing classes $C$ of $m$}
  \STATE $p_C\leftarrow\text{demand \code{final} parameter with fresh name, of type $C$}$
  \STATE make $p_C$ the $i$th parameter of $m'$
  \STATE $e\leftarrow\text{access to $C$\code{.this}}$
  \STATE insert $e$ as $i$th element into $xs$
  \STATE $qs\leftarrow \locked{p_C}::qs$
\ENDFOR
\STATE wrap body of $m'$ into \code{with(}$qs$\code{) \{}$\ldots$\code{\}}
\STATE set body of $m$ to \code{return}$\>\>\text{$\locked{v}$\code{.}$\locked{m}$\code{(}$xs$\code{)}}$\code{;}
\STATE $\refactoring{Insert Method}(T, m')$
\STATE eliminate \code{with} statement in $m'$
\STATE $\refactoring{Inline Temp}(v')$
\FORALL{$p_C$}
  \STATE $\refactoring{Remove Parameter}(p_C)\orelse\refactoring{Id}()$
\ENDFOR
\end{algorithmic}
\end{algorithm}

\subsection{\refactoring{Move Members}}
In order to move Field, static methods, and member types, we simply lock all references to them, as well as all names contained in them, and (for fields) the flow dependencies of their initialiser, and then move them inside the AST.

We include specification only for \refactoring{Move Static Method}; see Algorithm~\ref{alg:MoveStaticMethod}. 
They are all implemented in \sourcelink{Move/MoveMembers.jrag}.


\begin{algorithm}[p]
\caption{$\refactoring{Move Static Method}(m : \type{StaticMethod}, target : \type{Type})$}\label{alg:MoveStaticMethod}
\begin{algorithmic}[1]
\REQUIRE Java
\ENSURE Java $\cup$ locked names
\medskip
  \STATE $\util{lockAllNames}()$ in $m$
  \STATE $\util{lockNames}(\util{name}(m))$
  \STATE remove $m$ from $\util{hostType}(m)$ and insert it into the $target$
\end{algorithmic}
\end{algorithm}




\subsection{\refactoring{Promote Temp to Field}}
This refactoring turns a local variable into a field. Implemented in \sourcelink{PromoteTempToField/PromoteTempToField.jrag};
see Algorithms~\ref{alg:PromoteTemp},~\ref{alg:InsertField}.

\begin{algorithm}[p]
\caption{$\refactoring{Promote Temp to Field}(d : \type{LocalVar})$}
\label{alg:PromoteTemp}
\begin{algorithmic}[1]
\REQUIRE Java
\ENSURE Java $\cup$ locked dependencies
\medskip
\STATE $\refactoringNoExt{Split Declaration}(d)$
\STATE $d' \leftarrow \text{new \code{private} field of same type and name as $d$}$
\STATE make $d'$ \code{static} if $d$ is in static context
\STATE $\refactoringNoExt{Insert Field}(\util{hostType}(d), d')$
\FORALL{uses $u$ of $d$}
  \STATE lock $u$ onto $d'$
  \STATE lock reaching definitions of $u$
\ENDFOR
\STATE $\refactoring{Remove Decl}(d)$
\end{algorithmic}
\end{algorithm}

\begin{algorithm}[p]
\caption{$\refactoring{Insert Field}(T : \type{ClassOrInterface}, d : \type{Field})$}
\label{alg:InsertField}
\begin{algorithmic}[1]
\REQUIRE Java
\ENSURE Java $\cup$ locked names
\medskip
\STATE \assert $T$ has no local field with same name as $d$
\STATE \assert $d$ has no initialiser
\STATE \assert if $T$ is inner and $d$ is static, then $d$ is a constant
\STATE $\util{lockNames}(\util{name}(d))$
\STATE insert field $d$ into $T$
\end{algorithmic}
\end{algorithm}

\subsection{\refactoring{Pull Up}}
This refactoring pulls up a method $m$ from its host class $B$ to the super class $A$. Implemented in \sourcelink{PullUp/PullUpMethod.jrag}; see Algorithm~\ref{alg:PullUpMethod}.

\begin{algorithm}[p]
\caption{$\refactoring{Pull Up Method}(m : \type{Method})$}\label{alg:PullUpMethod}
\begin{algorithmic}[1]
\REQUIRE Java
\ENSURE Java $\cup$ locked names
\medskip
\STATE \assert the host type of $m$ $B$ is a non-library class
\STATE \assert the superclass $A$ of $B$ is also non-library
\STATE $m' \leftarrow \text{copy of $m$ with locked names}$
\STATE translate type variables in $m'$ from $B$ to $A$
\STATE $\refactoring{Insert Method}(A, m')$
\STATE remove $m$ from $B$
\end{algorithmic}
\end{algorithm}

TODO: explain translation of type variables; this is basically a right-inverse of the type variable substitution that happens when inheriting a method

Note that \refactoring{Insert Method} ensures that the inserted method is not called from anywhere.

\subsection{\refactoring{Push Down}}
This refactoring pushes a method down to all subclasses of its defining class. Implemented in \sourcelink{PushDown/PushDownMethod.jrag}; see Algorithms~\ref{alg:TriviallyOverride},~\ref{alg:RemoveMethod},~\ref{alg:MakeMethodAbstract},~\ref{alg:MakeTypeAbstract},~\ref{alg:PushDownVirtualMethod}.

Types that inherit a method $m$ include the host type of $m$.

\begin{algorithm}[p]
\caption{$\refactoring{Trivially Override}(B : \type{Type}, m : \type{VirtualMethod}) : \option{\type{MethodCall}}$}
\label{alg:TriviallyOverride}
\begin{algorithmic}[1]
\REQUIRE Java $\setminus$ implicit method modifiers
\ENSURE Java $+$ locked names, \code{return void}
\medskip
\STATE \assert $m$ is not \code{final}
\IF{$m$ not a member method of $B$}
  \RETURN \None
\ENDIF
\STATE $m' \leftarrow \text{copy of $m$ with locked names}$
\IF{$m$ is \code{abstract}}
  \STATE $\refactoring{Insert Method}(B, m')$
  \RETURN \None
\ELSE
  \STATE $xs \leftarrow \text{list of locked accesses to parameters of $m'$}$
  \STATE $c \leftarrow \text{\code{super.}$m$\code{(}$xs$\code{)}}$
  \STATE set body of $m'$ to \code{return}\xspace $c$\code{;}
  \STATE $\refactoring{Insert Method}(B, m')$
  \RETURN \Some{c}
\ENDIF
\end{algorithmic}
\end{algorithm}

\begin{algorithm}[p]
\caption{$\refactoring{Remove Method}(m : \type{Method})$}
\label{alg:RemoveMethod}
\begin{algorithmic}[1]
\REQUIRE Java
\ENSURE Java
\medskip
\STATE \assert $(\util{uses}(m)\cup\util{calls}(m))\setminus\util{below}(m)=\emptyset$
\STATE $o \leftarrow \{ m' \mid m <: m' \}$
\IF{$o\neq\emptyset\wedge\forall m'\in o.\text{$m'$ is abstract}$}
  \FORALL{$B$ in $\util{typesToMakeAbstract}(m)$}
    \STATE $\refactoring{Make Type Abstract}(B)$
  \ENDFOR
\ENDIF
\STATE remove $m$
\end{algorithmic}
\end{algorithm}

\begin{algorithm}[p]
\caption{$\refactoring{Make Method Abstract}(m : \type{Method})$}
\label{alg:MakeMethodAbstract}
\begin{algorithmic}[1]
\REQUIRE Java
\ENSURE Java
\medskip
\STATE \assert $m$ is not \code{native}, \code{static}, \code{private}, nor \code{final}
\STATE \assert there are no static calls to $m$ (e.g., \code{super}-call)
\FORALL{$B$ in $\util{typesToMakeAbstract}(m)$}
  \STATE $\refactoring{Make Type Abstract}(B)$
\ENDFOR
\STATE make $m$ \code{abstract}
\end{algorithmic}
\end{algorithm}

\begin{algorithm}[p]
\caption{$\refactoring{Make Type Abstract}(T : \type{Type})$}
\label{alg:MakeTypeAbstract}
\begin{algorithmic}[1]
\REQUIRE Java
\ENSURE Java
\medskip
\IF{$T$ is interface}
    \RETURN
\ENDIF
\STATE \assert $T$ is class and never instantiated
\STATE make $T$ \code{abstract}
\end{algorithmic}
\end{algorithm}

\begin{algorithm}[p]
\caption{$\refactoring{Push Down Virtual Method}(m : \type{VirtualMethod})$}
\label{alg:PushDownVirtualMethod}
\begin{algorithmic}[1]
\REQUIRE Java
\ENSURE Java $\cup$ locked names
\medskip
\FORALL{types $B<:\util{hostType}(m)$}
  \STATE $c \leftarrow \refactoringNoExt{Trivially Override}(B, m)$
  \IF{$c\neq\util{None}$}
    \STATE $\refactoring{Inline Method}(c)$
  \ENDIF
\ENDFOR
\STATE $\refactoring{Remove Method}(m)$
\STATE \qquad\orelse$\refactoring{Make Method Abstract}(m)$
\STATE \qquad\orelse$\refactoring{Id}()$
\end{algorithmic}
\end{algorithm}

\subsection{\refactoring{Rename}}
This family of refactorings is used for renaming named program entities. Implemented in \sourcelink{Renaming/}.

\begin{algorithm}[p]
\caption{$\refactoring{Rename Field}(f : \type{Field}, n : \type{Name})$}\label{alg:RenameField}
\begin{algorithmic}[1]
\REQUIRE Java
\ENSURE Java $\cup$ locked names
\medskip
\STATE \assert $n$ is a valid name
\STATE \assert host type of $f$ contains no other field of name $n$
\STATE $\util{lockNames}(\{n, \util{name}(f)\})$
\STATE set name of $f$ to $n$
\end{algorithmic}
\end{algorithm}

Refactoring \refactoring{Rename Field} (Algorithm \ref{alg:RenameField}) changes the name of a field $f$ to $n$. It ensures that $n$ is indeed a valid name and that the host type of $f$ contains no other field called $n$. It then globally locks all accesses to variables, types, or packages named either $n$ or $\util{name}(f)$, and changes the name of $f$ to $n$.

\begin{algorithm}[p]
\caption{$\refactoring{Rename Local}(v : \type{Local}, n : \type{Name})$}\label{alg:RenameLocal}
\begin{algorithmic}[1]
\REQUIRE Java
\ENSURE Java $\cup$ locked names
\medskip
\STATE \assert $n$ is a valid name
\STATE \assert scope of $v$ does not intersect scope of any other \type{Local} named $n$
\STATE $\util{lockNames}(\util{block}(v), \{n, \util{name}(f)\})$
\STATE set name of $v$ to $n$
\end{algorithmic}
\end{algorithm}

Refactoring \refactoring{Rename Local} (Algorithm \ref{alg:RenameLocal}) changes the name of a local variable or parameter $v$ to $n$. It ensures that $n$ is indeed a valid name and that the renaming $v$ to $n$ will not violate the rule that scopes of local variables of the same name cannot be nested. It then again locks all accesses to variables, types, or packages named either $n$ or $\util{name}(v)$, but only within the enclosing block of $v$, and changes the name of $v$ to $n$.

\begin{algorithm}[p]
\caption{$\refactoring{Rename Method}(m : \type{Method}, n : \type{Name})$}\label{alg:RenameMethod}
\begin{algorithmic}[1]
\REQUIRE Java
\ENSURE Java $\cup$ locked names, locked overriding
\medskip
\STATE \assert $n$ is a valid name
\STATE $\util{lockMethodNames}(\{\util{name}(m), n\})$
\STATE $\util{lockOverriding}(\{\util{name}(m), n\})$
\FORALL{$m'$ such that $\exists m''.m<:^*m'' \wedge m'<:^*m''$}
  \STATE \assert $m'$ is not native
  \STATE $s\leftarrow\text{signature of $m'$ after renaming}$
  \STATE \assert host type of $m'$ contains no local method of signature $s$
  \STATE \assert $m'$ can override or hide any ancestor method of signature $s$
  \STATE \assert $m'$ can be overridden or hidden by any descendant method of signature $s$
  \STATE set name of $m'$ to $n$
  \STATE remove any static import of $m'$ if it would become vacuous
\ENDFOR
\end{algorithmic}
\end{algorithm}

Refactoring \refactoring{Rename Method} (Algorithm \ref{alg:RenameMethod}) changes the name of a method $m$ to $n$. It ensures that $n$ is a valid name, then locks all calls to methods of name $\util{name}(m)$ or $n$, and their overriding dependencies. Now it changes the names of all methods $m'$ related to $m$ (\emph{i.e.}, such that $m$ and $m'$ both transitively override the same method), checking that the resulting program will be well-formed: in particular, there cannot be another local method with the same signature, and any methods that the renamed $m'$ would override or hide must, in fact, be overridable or hidable by $m'$, and vice versa for methods that would override or hide $m'$. If there is a static import that only imports $m'$ (and not also another static member of the surrounding class), then remove that import. We could, of course, try to adjust it, but changing imports is a tricky business.

\begin{algorithm}[p]
\caption{$\refactoring{Rename Type}(T : \type{Type}, n : \type{Name})$}\label{alg:RenameType}
\begin{algorithmic}[1]
\REQUIRE Java
\ENSURE Java $\cup$ locked names
\medskip
\STATE \assert $n$ is a valid name
\STATE \assert no native method is nested in $T$
\STATE \assert there is no nesting or enclosing type of name $n$
\STATE \assert if $T$ is a toplevel type, there is no other toplevel type $n$ in the enclosing package, and it has no subpackage of name $n$
\STATE \assert if $T$ is a type parameter, there is no type parameter of name $n$ in the parameter list where it occurs
\STATE $\util{lockNames}(\{\util{name}(T), n\})$
\STATE set name of $T$ to $n$
\STATE set names of constructors of $T$ to $n$
\STATE if $T$ is public, change the name of its compilation unit to match
\STATE remove any single type import declaration of $T$ that would clash with a visible type or with another import declaration
\STATE remove any static import of $T$ if it would become vacuous
\end{algorithmic}
\end{algorithm}

Refactoring \refactoring{Rename Type} (Algorithm \ref{alg:RenameType}) changes the name of a type $T$ to $n$.  It is fairly straightforward, except for the well-formedness checks and the treatment of import declarations.



\begin{algorithm}[p]
\caption{$\refactoring{Rename Package}(P : \type{Package}, n : \type{Name})$}\label{alg:RenamePackage}
\begin{algorithmic}[1]
\REQUIRE Java
\ENSURE Java $\cup$ locked names
\medskip
\STATE \assert $n$ is a valid name
\STATE \assert there is no package with name $n$
\STATE $\util{lockAllPackageAccesses}()$
\FORALL{$package$ in \{packages with name starting with $\util{name}(P)\}$}
  \STATE $\refactoring{Rename Package}(package, \util{name}(package).\util{replaceFirst}(\util{name}(P), n))$
\ENDFOR
\STATE set name of $P$ to $n$
\end{algorithmic}
\end{algorithm}

Refactoring \refactoring{Rename Package} (Algorithm \ref{alg:RenamePackage}) changes the name of a package $P$ to $n$ and renames also all subpackages.



\subsection{\refactoring{Self-Encapsulate Field}}
This refactoring makes a field private, rerouting all accesses to it through getter and setter methods. Implemented in \sourcelink{SelfEncapsulateField/SelfEncapsulateField.jrag}; see Algorithm~\ref{alg:Self-EncapsulateField}.

\begin{algorithm}[p]
\caption{$\refactoring{Self-Encapsulate Field}(f : \type{Field})$}\label{alg:Self-EncapsulateField}
\begin{algorithmic}[1]
\REQUIRE Java $\setminus$ abbreviated assignments
\ENSURE Java $\cup$ locked names
\medskip
\STATE create getter method $g$ for $f$
\STATE if $f$ is not \code{final}, create setter method $s$ for it
\FORALL{all uses $u$ of $f$ and its substituted copies}
  \IF{$u\not\in\util{below}(g)\cup\util{below}(s)$}
    \IF{$u$ is an rvalue}
      \STATE replace $u$ with locked access to $g$
    \ELSE
      \IF{$f$ is not \code{final}}
        \STATE $q \leftarrow \text{qualifier of $u$, if any}$
        \STATE $r \leftarrow \text{RHS of assignment for which $u$ is LHS}$
        \STATE replace $u$ with locked access to $s$ on argument $r$, qualified with $q$ if applicable
      \ENDIF
    \ENDIF
  \ENDIF
\ENDFOR
\end{algorithmic}
\end{algorithm}

By ``abbreviated assignment'' we mean \code{x+=y} and friends, as well as increment and decrement expressions. The language restriction tries to expand these into normal assignments, but may fail if the data flow is too complicated. If it succeeds, every lvalue will appear on the left hand side of a (simple) assignment.

Note that even when $f$ is final there may still be assignments to $f$ from within constructors; we cannot encapsulate these assignments, so we skip them.


\clearpage

\section{Node Types}
See Fig.~\ref{fig:node types}. We also use the non-node type \type{Name} to represent names.

\begin{figure}[hb]
\begin{center}
\begin{tabular}{|l|p{5cm}|}
\hline
\textbf{Node Type} & \textbf{Description} \\ \hline\hline
\type{ClassOrInterface} & either a class or an interface; is a \type{Type} \\
\type{Field} & field declaration \\
\type{LocalVar} & local variable declaration \\
\type{MemberType} & type declared inside another type; is a \type{Type} \\
\type{Method} & method declaration \\
\type{MethodCall} & method call \\
\type{Package} & package \\
\type{Type} & type declaration \\
\type{VirtualMethod} & non-\code{private} instance method; is a \type{Method} \\
\hline
\end{tabular}
\end{center}
\caption{Node Types}
\label{fig:node types}
\end{figure}



\section{Utility Functions}
See Fig.~\ref{fig:utility}.

\begin{figure}[hb]
\begin{center}
\begin{tabular}{|l|p{5.3cm}|}
\hline
\textbf{Name} & \textbf{Description} \\ \hline\hline
$\util{below}(n)$ & returns the set of all nodes below $n$ in the syntax tree \\
$\util{calls}(m)$ & returns all calls that may dynamically resolve to method $m$; can be a conservative over-approximation \\
$\util{hostPkg}(e)$ & returns the package of the compilation unit containing $e$ \\
$\util{hostType}(e)$ & returns the closest enclosing type declaration around $e$ \\
$\util{lockMethodCalls}(n)$ & locks all calls to methods named $n$ anywhere in the program \\
$\util{lockNames}(n)$ & locks all names anywhere in the program that refer to a declaration with name $n$ \\
$\util{name}(e)$ & returns the name of program entity $e$ \\
$\util{uses}(m)$ & returns all calls that statically bind to method $m$ \\
\hline
\end{tabular}
\end{center}
\caption{Utility Functions}
\label{fig:utility}
\end{figure}


\clearpage

\listofalgorithms
\end{document}
